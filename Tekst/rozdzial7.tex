
\chapter{Interfejs u�ytkownika}

W poprzednim rodziale opisano wszystkie dzia�ania, kt�re podj�to w
celu dokonania pomiaru przy�pieszenia i k�ta obrotu uk�adu. Dane przes�ane
do komputera nie mog� by� jedynie wy�wietlone po stronie komputera
w postaci ci�gu znak�w, gdy� to praktycznie uniemo�liwia�oby analiz�
otrzymywanych danych. Z tego wzgl�du konieczne jest odpowiednie przekszta�cenie
otrzymanych danych w spos�b, kt�ry umo�liwi zobrazowanie zmierzonych
przyspiesze� b�d� obrot�w. W tym celu przygotowano aplikacj�, kt�rej
zadaniem jest odbi�r i wy�wietlenie w formie graficznej wynik�w. Wykorzystano
do tego platform� .NET i j�zyk C\#. Powodem wyboru tej technologii
by�a wzgl�dna prostota i mnogo�� klas i technik, kt�re umo�liwia�y
osi�gni�cie postawionego celu.


\section{Odebranie danych z portu szeregowego}
